% DFF template for Latex and A4 paper.
% 12pt Times New Roman on 1.15 line spacing and 2 cm margins.

% ----------------------------------------------------------------------

% Either format with 
%    pdflatex projectdescription.tex
% Or if you use dvips and ps2pdf, remember to specify A4 paper:
%    latex  projectdescription
%    dvips  -ta4 projectdescription -o projectdescription.ps
%    ps2pdf -sPAPERSIZE=a4 projectdescription.ps

% -----------------------------------------------------------------------

\documentclass[fleqn,12pt]{article}
%Do not change this geometry settings
\usepackage[a4paper,top=2cm,bottom=2cm,left=2cm,right=2cm]{geometry}
\usepackage{times}
\usepackage[english]{babel}
\addto\captionsenglish{% Replace "english" with the language you use
  \renewcommand{\contentsname}%
    {Table of Contents}%
}
\usepackage[utf8]{inputenc}
\usepackage[T1]{fontenc}
\usepackage[numbers,sort&compress]{natbib}
%\usepackage[pdftex]{graphicx}
\usepackage{hyperref}
\usepackage{xcolor}
\usepackage{sectsty}
\usepackage{caption}
\usepackage{subfig}

\sectionfont{\fontsize{12}{15}\selectfont}
\hypersetup{
    colorlinks=true,
    linkcolor = black,
    citecolor =blue,
    filecolor=blue,      
    urlcolor=blue,
}
\usepackage{graphicx}         % For PDF figure
% \usepackage[dvips]{graphicx}  % For EPS figures, using dvips + ps2pdf
\setcounter{secnumdepth}{0}
\begin{document}

\setlength{\baselineskip}{1.15\baselineskip}


% ---------------------------------------------------------------------

\begin{center}
  {\Huge Reflections on Data Science 2023}\\[2ex]
  {Lukas Rasocha, Gergo Gyori, Katalin Literati-Dobos, Ludek Cizinsky}\\
  [2ex]
  {\{lukr,gegy,klit,luci\}@itu.dk}\\[2ex]
\end{center}

\tableofcontents

% ----------------------------------------------------------------------

%\parskip=3mm

%\noindent

\parindent=20pt 
\parskip=0mm

\newpage


\section{\#3. Lead author: [Lukas Rasocha], Calling BS on \href{https://twitter.com/JunkScience/status/1613724250011242497?s=20}{"Steve Milloy's tweet claiming that global warming due to C02 emmission is a hoax."} (January, 2023)} 

Steve Milloy, the author of the \href{https://twitter.com/JunkScience/status/1613724250011242497?s=20}{tweet}, argues that global warming caused by CO2 emissions is a hoax, 
based on a figure comparing yearly temperature to the 20th-century average. He includes a regression line to support his claim of a cooling trend and a declining temperature rate. 
However, Milloy's \textbf{visualization is misleading}. Figure \ref{fig:entry3} presents both Milloy's figure and the original figure from the \href{https://www.noaa.gov/news/2022-was-worlds-6th-warmest-year-on-record}{NOAA's press release}, 
contradicting his assertions.

The original figure intends to demonstrate that it has been 46 years since the average yearly temperature was cooler than the 20th-century average. 
Furthermore, it highlights that the years 2014 to 2022 rank among the ten warmest years. Thus, the author distorts the visualization by selectively zooming in on a small portion of the original figure, obscuring the overall long-term warming trend. 
Additionally, Professor of geosciences at Princeton University Gabriel Vecchi explains in one of the \href{https://www.factcheck.org/2023/01/scicheck-viral-tweet-misrepresents-noaa-report-on-rising-global-temperature/}{articles} 
that climate patterns such as \href{https://oceanservice.noaa.gov/facts/ninonina.html}{El Niño and La Niña} can cause temporary surface warming and cooling, resulting in fluctuations in the warming rate over short periods. 
Consequently, cherry-picking specific timeframes to portray a trend is an invalid approach.

Global warming is an ongoing and significant issue, with substantial efforts already undertaken. 
However, much work remains, and politicians play a critical role in driving change to mitigate global warming. 
To facilitate the implementation of measures by politicians, it is essential to ensure the public is well-informed. 
Unfortunately, tweets like the one by Steve Milloy only deepen divisions among people's opinions instead of fostering unity. 
As a consequence, progress towards achieving zero CO2 emissions is hindered, slowing down potential advancements.

\begin{figure}[h!]
    \centering
    \begin{tabular}{cc}
        \subfloat[Steve Milloy's figure used in his \href{https://twitter.com/JunkScience/status/1613724250011242497?s=20}{tweet}.]{\includegraphics[width = 3in]{figures/e1_fig1.png}} &
        \subfloat[Original figure from \href{https://www.noaa.gov/news/2022-was-worlds-6th-warmest-year-on-record}{NOAA's press release}.]{\includegraphics[width = 3in]{figures/e2_fig2.png}} \\
    \end{tabular}
    \caption{Supporting evidence for the third entry.}
    \label{fig:entry3}
\end{figure}

\newpage

\end{document}