% DFF template for Latex and A4 paper.
% 12pt Times New Roman on 1.15 line spacing and 2 cm margins.

% ----------------------------------------------------------------------

% Either format with 
%    pdflatex projectdescription.tex
% Or if you use dvips and ps2pdf, remember to specify A4 paper:
%    latex  projectdescription
%    dvips  -ta4 projectdescription -o projectdescription.ps
%    ps2pdf -sPAPERSIZE=a4 projectdescription.ps

% -----------------------------------------------------------------------

\documentclass[fleqn,12pt]{article}
%Do not change this geometry settings
\usepackage[a4paper,top=2cm,bottom=2cm,left=2cm,right=2cm]{geometry}
\usepackage{times}
\usepackage[english]{babel}
\addto\captionsenglish{% Replace "english" with the language you use
  \renewcommand{\contentsname}%
    {Table of Contents}%
}
\usepackage[utf8]{inputenc}
\usepackage[T1]{fontenc}
\usepackage[numbers,sort&compress]{natbib}
%\usepackage[pdftex]{graphicx}
\usepackage{hyperref}
\usepackage{xcolor}
\usepackage{sectsty}
\usepackage{caption}
\usepackage{subfig}

\sectionfont{\fontsize{12}{15}\selectfont}
\hypersetup{
    colorlinks=true,
    linkcolor = black,
    citecolor =blue,
    filecolor=blue,      
    urlcolor=blue,
}
\usepackage{graphicx}         % For PDF figure
% \usepackage[dvips]{graphicx}  % For EPS figures, using dvips + ps2pdf
\setcounter{secnumdepth}{0}
\begin{document}

\setlength{\baselineskip}{1.15\baselineskip}


% ---------------------------------------------------------------------

\begin{center}
  {\Huge Reflections on Data Science 2023}\\[2ex]
  {Lukas Rasocha, Gergo Gyori, Katalin Literati-Dobos, Ludek Cizinsky}\\
  [2ex]
  {\{lukr,gegy,klit,luci\}@itu.dk}\\[2ex]
\end{center}

\tableofcontents

% ----------------------------------------------------------------------

%\parskip=3mm

%\noindent

\parindent=20pt 
\parskip=0mm

\newpage

%%%%%%%%%%%%%%%%%%%%%%%%%%%%%%% ENTRY 3 %%%%%%%%%%%%%%%%%%%%%%%%%%%%%%%%
% ----------------------------------------------------------------------
\section{\#3. Lead author: [Lukas Rasocha], Calling BS on \href{https://twitter.com/JunkScience/status/1613724250011242497?s=20}{"Steve Milloy's tweet claiming that global warming due to C02 emmission is a hoax."} (January, 2023)} 

Steve Milloy, the author of the \href{https://twitter.com/JunkScience/status/1613724250011242497?s=20}{tweet}, argues that global warming caused by CO2 emissions is a hoax, 
based on a figure comparing yearly temperature to the 20th-century average. He includes a regression line to support his claim of a cooling trend and a declining temperature rate. 
However, Milloy's \textbf{visualization is misleading}. Figure \ref{fig:entry3} presents both Milloy's figure and the original figure from the \href{https://www.noaa.gov/news/2022-was-worlds-6th-warmest-year-on-record}{NOAA's press release}, 
contradicting his assertions.

The original figure intends to demonstrate that it has been 46 years since the average yearly temperature was cooler than the 20th-century average. 
Furthermore, it highlights that the years 2014 to 2022 rank among the ten warmest years. Thus, the author distorts the visualization by selectively zooming in on a small portion of the original figure, obscuring the overall long-term warming trend. 
Additionally, Professor of geosciences at Princeton University Gabriel Vecchi explains in one of the \href{https://www.factcheck.org/2023/01/scicheck-viral-tweet-misrepresents-noaa-report-on-rising-global-temperature/}{articles} 
that climate patterns such as \href{https://oceanservice.noaa.gov/facts/ninonina.html}{El Niño and La Niña} can cause temporary surface warming and cooling, resulting in fluctuations in the warming rate over short periods. 
Consequently, cherry-picking specific timeframes to portray a trend is an invalid approach.

Global warming is an ongoing and significant issue, with substantial efforts already undertaken. 
However, much work remains, and politicians play a critical role in driving change to mitigate global warming. 
To facilitate the implementation of measures by politicians, it is essential to ensure the public is well-informed. 
Unfortunately, tweets like the one by Steve Milloy only deepen divisions among people's opinions instead of fostering unity. 
As a consequence, progress towards achieving zero CO2 emissions is hindered, slowing down potential advancements.

\begin{figure}[h!]
    \centering
    \begin{tabular}{cc}
        \subfloat[Steve Milloy's figure used in his \href{https://twitter.com/JunkScience/status/1613724250011242497?s=20}{tweet}.]{\includegraphics[width = 3in]{figures/e1_fig1.png}} &
        \subfloat[Original figure from \href{https://www.noaa.gov/news/2022-was-worlds-6th-warmest-year-on-record}{NOAA's press release}.]{\includegraphics[width = 3in]{figures/e2_fig2.png}} \\
    \end{tabular}
    \caption{Supporting evidence for the third entry.}
    \label{fig:entry3}
\end{figure}

\newpage

%%%%%%%%%%%%%%%%%%%%%%%%%%%%%%% ENTRY 4 %%%%%%%%%%%%%%%%%%%%%%%%%%%%%%%%
% ----------------------------------------------------------------------
\section{\#4. Lead author: [Lukas Rasocha], Calling BS on \href{https://twitter.com/drsimonegold/status/1610361145294000131?s=20}{"Dr. Simone Gold's tweet claiming 'The same number of athletes died in the last TWO years as 
compared to a prior 38 years.'"} (January, 2023)}
In her Twitter thread, Dr. Simone Gold, an experienced emergency physician, begins by stating that prior to 2020, athletes being incapacitated or dropping dead was not a common occurrence. 
However, she expresses concern that such incidents are now happening with alarming frequency. To support her claim, Dr. Gold backs her claim by two statistics coming from the \href{https://www.ncbi.nlm.nih.gov/pmc/articles/PMC9877705/}{letter} 
to the editor of an immunology journal. The first statistic (\href{https://academic.oup.com/eurjpc/article/13/6/859/5932831}{source}) states that between 1966 and 2004 (a span of 38 years), a total of 1101 athletes died due to various heart conditions. 
The second statistic, sourced from another article (\href{https://goodsciencing.com/covid/athletes-suffer-cardiac-arrest-die-after-covid-shot/}{source}), indicates that from 2021 until the beginning of 2023, the same number of athletes died. 
To summarise, I call BS on Dr. Gold claim that \textit{The same number of athletes died in the last TWO years as compared to a prior 38 years.} due to \textbf{unfair comparison} being made since the data collection methodology of the two sources differs in several aspects.

Firstly, the first source focuses specifically on athletes under the age of 35, whereas the second source encompasses athletes of all age groups. 
Secondly, there are disparities in how the two sources define an athlete. The definition provided in the second source is more ambiguous and encompasses a 
broader spectrum of individuals who engage in athletic activities. Thirdly, it is important to note that the first source only includes athletes who died from 
various heart conditions, while the second source takes into account all types of deaths among athletes. 
Consequently, drawing meaningful conclusions based on the comparison of these two datasets is not valid due to the inherent unfairness in the comparison.
Although Dr. Gold does not explicitly state it in her tweet, the context, such as a screenshot of the letter, 
suggests that she is attempting to link the increased death rate among athletes to vaccination. 
Such a tweet supports the opinion that vaccines are dangerous and advocates for avoiding them. 
However, promoting vaccine hesitancy in this manner can hinder the progress of recovering from the global pandemic and returning to the "normal world".

\end{document}