% DFF template for Latex and A4 paper.
% 12pt Times New Roman on 1.5 line spacing and 2 cm margins.

% ----------------------------------------------------------------------

% Either format with 
%    pdflatex projectdescription.tex
% Or if you use dvips and ps2pdf, remember to specify A4 paper:
%    latex  projectdescription
%    dvips  -ta4 projectdescription -o projectdescription.ps
%    ps2pdf -sPAPERSIZE=a4 projectdescription.ps

% ----------------------------------------------------------------------

\documentclass[fleqn,12pt]{article}
%Do not change this geometry settings
\usepackage[a4paper,top=2cm,bottom=2cm,left=2cm,right=2cm]{geometry}
\usepackage{times}
\usepackage[english]{babel}
\usepackage[utf8]{inputenc}
\usepackage[T1]{fontenc}
\usepackage[numbers,sort&compress]{natbib}
\usepackage[pdftex]{graphicx}
\usepackage{hyperref}
%\usepackage{titlesec}
\hypersetup{
    colorlinks=true,
    linkcolor = black,
    citecolor =blue,
    filecolor=magenta,      
    urlcolor=cyan,
}

% \usepackage{graphicx}         % For PDF figures
% \usepackage[dvips]{graphicx}  % For EPS figures, using dvips + ps2pdf

\begin{document}
% Empirically this seems to match MS Word's idea of 1.5 line spacing.
% DO NOT CHANGE
\setlength{\baselineskip}{1.44\baselineskip}


% ----------------------------------------------------------------------
% Enter the topic of the project and your names 

\begin{flushleft}
  {\large Gergo Gyori (gegy@itu.dk), BSc Data Science \\
  Katalin Literati-Dobos (klit@itu.dk), BSc Data Science \\
  Ludek Cizinsky (luci@itu.dk), BSc Data Science \\
  Lukas Rasocha (lukr@itu.dk), BSc Data Science \\}
 \end{flushleft}
 
\begin{center}
  {\Large Reflections on Data Science 2023}\\[5ex]
  {\Large Social Study: Herding effect on Reddit}\\[5ex]

 \end{center}
 

% ----------------------------------------------------------------------
% Delete the instruction 

%\parskip=3mm

%\noindent

\parindent=20pt 
\parskip=0mm

\section{Introduction}
Posts that achieve popularity on social media platforms, 
like Reddit, often outperform average posts significantly, 
which indicates a qualitative difference between popular posts 
and the rest. With this phenomenon an interesting question emerges:
is the success of a post attributable to its inherent quality,
or is it influenced by the herd effect? This social psychological 
behaviour that leads individuals to perceive an action as the 
appropriate course simply because it's what "everyone else" 
seems to be doing. This effect has already been documented in various studies such as \cite{muchnik} \cite{salganik}.
In this study we aim to investigate the herd effect on Reddit, more
concretely, we want to find out to what extent does an initial upvote to a post
influence its future score?


\section{Experiment implementation}

\section{Results and discussion}

Discuss your findings in light of the theories and concepts covered in the course (e.g., causality, experimental designs,
extraneous variables, social influence, reproducibility, etc.)

Since we learned a bit too late about the vote fuzzing our experiment only recorded the overall score of the posts, while not looking at changing number of comments
which might have been a better indication (no comment count fuzzing).
\section{Conclusion}

% ----------------------------------------------------------------------
% References 

 \newpage 
%\section*{References}
\bibliographystyle{unsrt}
\bibliography{yourbib}

\end{document}