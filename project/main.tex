% DFF template for Latex and A4 paper.
% 12pt Times New Roman on 1.5 line spacing and 2 cm margins.

% ----------------------------------------------------------------------

% Either format with 
%    pdflatex projectdescription.tex
% Or if you use dvips and ps2pdf, remember to specify A4 paper:
%    latex  projectdescription
%    dvips  -ta4 projectdescription -o projectdescription.ps
%    ps2pdf -sPAPERSIZE=a4 projectdescription.ps

% ----------------------------------------------------------------------

\documentclass[fleqn,12pt]{article}
%Do not change this geometry settings
\usepackage[a4paper,top=2cm,bottom=2cm,left=2cm,right=2cm]{geometry}
\usepackage{times}
\usepackage[english]{babel}
\usepackage[utf8]{inputenc}
\usepackage[T1]{fontenc}
\usepackage[numbers,sort&compress]{natbib}
\usepackage[pdftex]{graphicx}
\usepackage{hyperref}
%\usepackage{titlesec}
\hypersetup{
    colorlinks=true,
    linkcolor = black,
    citecolor =blue,
    filecolor=magenta,      
    urlcolor=cyan,
}

% \usepackage{graphicx}         % For PDF figures
% \usepackage[dvips]{graphicx}  % For EPS figures, using dvips + ps2pdf

\begin{document}
% Empirically this seems to match MS Word's idea of 1.5 line spacing.
% DO NOT CHANGE
\setlength{\baselineskip}{1.44\baselineskip}


% ----------------------------------------------------------------------
% Enter the topic of the project and your names 

\begin{flushleft}
  {\large Gergo Gyori (gegy@itu.dk), BSc Data Science \\
  Katalin Literati-Dobos (klit@itu.dk), BSc Data Science \\
  Ludek Cizinsky (luci@itu.dk), BSc Data Science \\
  Lukas Rasocha (lukr@itu.dk), BSc Data Science \\}
 \end{flushleft}
 
\begin{center}
  {\Large Reflections on Data Science 2023}\\[5ex]
  {\Large Social Study: Herding effect on Reddit}\\[5ex]

 \end{center}
 

% ----------------------------------------------------------------------
% Delete the instruction 

%\parskip=3mm

%\noindent

\parindent=20pt 
\parskip=0mm

\section{Introduction}

This is a reference \cite{bergstrom2020calling}.

Popular posts on social media are many times more successful than the average post
 This suggests that the best alternatives are intrinsically/qualitatively different from the
rest  yet predicting this is difficult!
 Is post success due to intrinsic differences in quality or due to the herd bias?

 Herd effect
 is a psychological phenomenon in which
people rationalise that a course of action
is the right one because 'everybody else'
is doing it
\section{Experiment implementation}

\section{Results and discussion}

Discuss your findings in light of the theories and concepts covered in the course (e.g., causality, experimental designs,
extraneous variables, social influence, reproducibility, etc.)
\section{Conclusion}

% ----------------------------------------------------------------------
% References 

 \newpage 
%\section*{References}
\bibliographystyle{unsrt}
\bibliography{yourbib}

\end{document}